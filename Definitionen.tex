%! Author = engel
%! Date = 04.06.2024

% Preamble
\documentclass[11pt]{article}

% Packages
\usepackage{amsmath}
\usepackage{amsfonts}
\usepackage{amssymb}
\usepackage{hyperref}
\usepackage{stmaryrd}
\usepackage{mathtools}
\usepackage{enumerate}

% Document
\begin{document}

    \section{Beschränkungen}\label{sec:beschrankungen}

    \subsection{Obere Schranke}\label{subsec:obere-schranke}
    \begin{align*}
        \exists S:\forall x\in M: x\leq S
    \end{align*}

    \subsection{Untere Schranke}\label{subsec:untere-schranke}
    \begin{align*}
        \exists S:\forall x\in M: x\geq S
    \end{align*}

    \subsection{Maximum}\label{subsec:maximum}
    Das Maximum $a$ ist genau dann ein Maximum der Menge $M$, wenn gilt:
    \begin{align*}
        a\in M: \forall x\in M: x\leq a
    \end{align*}
    Das heißt aber im Umkehrschluss, dass es nicht zwingend ein Maximum geben muss.
    So zum Beispiel $M=[-1,1)$, da die $1$ nicht in der Menge enthalten ist, und es keine Näherung zu $1$ gibt.

    \subsection{Minimum}\label{subsec:minimum}
    Das Minimum $a$ ist genau dann ein Minimum der Menge $M$, wenn gilt:
    \begin{align*}
        a\in M: \forall x\in M: x\geq a
    \end{align*}
    Das heißt aber im Umkehrschluss, dass es nicht zwingend ein Minimum geben muss.
    So zum Beispiel $M=(-1,1]$, da die $-1$ nicht in der Menge enthalten ist, und es keine Näherung zu $-1$ gibt.

    \subsection{Supremum}\label{subsec:supremum}
    Das Supremum ist die kleinste obere Grenze und damit der größte Punkt in der Menge $M$.\\
    Anders ausgedrückt, die kleinste obere Schranke.

    \subsection{Infimum}\label{subsec:infimum}
    Das Supremum ist die größte untere Grenze und damit der kleinste Punkt in der Menge $M$.\\
    Anders ausgedrückt, die größte untere Schranke.

    \subsection{Unterschied}\label{subsec:unterschied}
    Dabei gilt der Unterschied zwischen dem \textbf{Supremum} und dem \textbf{Maximum}, beziehungsweise dem \textbf{Infimum} und dem \textbf{Minimum}, dass
    das \textbf{Maxi-} und dem \textbf{Minimum} zwingen in der Menge $M$ enthalten sein müssen, während das \textbf{Supremum} und das \textbf{Infimum} gegebenenfalls auch außerhalb der Menge $M$ liegen können.

    \section{Algebraische Funktion}\label{sec:gruppen}
    Eine algebraische Struktur ist ein Konstrukt mit der Syntax $(M, \circ)$ mit der Menge $M$ und der Operation $\circ$.
Solche Strukturen haben verschiedene Eigenschaften.
Dazu zählen:
\begin{enumerate}
    \item \textbf{Kommutativgesetz:} $\forall a, b\in M:a\circ b=b\circ a$
    \item \textbf{Assoziativgesetz:} $\forall a, b, c\in M:(a\circ b)\circ c=a\circ (b\circ c)$
    \item \textbf{Existenzsatz:} $\forall a, b\in M: \exists x\in M: a\circ x=b$\\
    bzw. $\forall a, b\in M: \exists x: x\circ a=b$
    \item \textbf{Eindeutigkeitssatz:} $\forall a, b\in M:\forall x_1, x_2\in M : a\circ x_1 =b \wedge a\circ x_2=b\rightarrow x_1=x_2$\\
    bzw. $\forall a, b\in M:\forall x_1, x_2\in M : x_1\circ a =b \wedge x_2\circ a=b\rightarrow x_1=x_2$
\end{enumerate}
\subsection{Gruppe}
Eine algebraische Struktur ($G$, $\circ$) heißt genau dann Gruppe, wenn folgende Axiome erfüllt sind:
\begin{enumerate}
    \item \textbf{Existenzsatz}
    \item \textbf{Assoziativgesetz}
    \item \textbf{Linksneutrales Element:} In $G$ gibt es ein linksneutrales Element: $\exists e\in G: \forall a\in G: a\circ e = a$
    \item \textbf{Inverses Element:} In $G$ gibt es zu jedem Element ein Inverses Element: $\forall a\in G:\exists a'\in G: a\circ a'=a'\circ a=e$
\end{enumerate}
Die Gruppe heißt zusätzlich Abel'sche Gruppe, wenn das \textbf{Kommutativgesetz} erfüllt ist.
\subsection{Elementare Gruppeneigenschaften}\label{subsec:elementare-gruppeneigenschaften}
\begin{enumerate}
    \item Sei $e\in G$ linksneutrales Element und $a,a'\in G$, sodass $a'\circ a=e$.
    Dann gilt auch $a\circ a'=e$.
    Man nennt $a'$ dann nur noch ein inverses zu $a$.
    \item Sei $e\in G$ linksneutrales Element.
    Dann gilt $a\circ e=a$ für alle $a\in G$ und man nennt $e$ dann ein neutrales Element.
    \item Es gibt genau ein neutrales Element $e\in G$.
    \item Zu jedem $a\in G$ gibt es genau ein inverses Element $a'\in G$ (oft mit $a^-1$ bezeichnet).
    \item $(a^{-1})^{-1}=a$.
    \item $(a\circ b)^{-1}=b^{-1}\circ a^{-1}$.
\end{enumerate}

\subsection{Gruppeneigenschaften nachweisen}\label{subsec:gruppeneigenschaften-nachweisen}
\begin{enumerate}
    \item Wählen wir $a''$ mit $a''\circ a'=e$.
    Dann gilt: $a\circ a'=e\circ a\circ a'=a''\circ a'\circ a\circ a'=a''\circ e\circ a'=a''\circ a'=e$
    \item $a\circ e=a\circ(a'\circ a)=(a\circ a')\circ a=e\circ a=a$
    \item Wir nehmen an, es gäbe zwei neutrale Elemente von $G$ ($e,e^*\in G$).
    Dann ist jedoch $e^*=e\circ e^*=e$.
    Die beiden neutralen Elemente sind also zwangsläufig gleich.
    \item Wir nehmen an, es gäbe ein zweites inverses Element $a^*$ neben $a'$ zu $a$ in $G$.
    Dann muss jedoch $a^*=a'$ sein: $a^*=a^*\circ e=a^*\circ (a\circ a')=(a^*\circ a)\circ a'=e\circ a'=a'$
    \item Es gilt $(a^{-1})^{-1}\circ a^{-1}=e$.
    Es gilt aber auch  $a\circ a^{-1}=e$.
    Zu $a^{-1}$ gibt es jedoch nur ein inverses Element (nach der vierten Eigenschaft der Gruppen).
    Also ist $(a^{-1})^{-1}=a$.
    \item Es gilt $(b^{-1}\circ a^{-1})\circ(a\circ b)=b^{-1}\circ(a^{-1}\circ a)\circ b=b^{-1}\circ e\circ b=b^{-1}\circ b=e$.
    Somit ist $(b^{-1}\circ a^{-1})$ ds inverse Element zu $a\circ b$.
    Also ist $(a\circ b)^{-1}=b^{-1}\circ a^{-1}$.
\end{enumerate}
\subsection{Satz von Lagrange}
$(G, \circ)$ Gruppe, $H\subseteq G$ eine Untergruppe.
$ord(H)|ord(G)$
$\forall g\in G: ord(g)|ord(G)$
$g_1\equiv g_2 \leftrightarrow g_2^{-1}g_1\in H$
\subsection{Satz von Fermat}
Primzahl $p$, $0<a\leq p-1$
\begin{align}
    a^{p-1}\equiv_{p}1\\
    a^p\equiv_{p}a
\end{align}
\begin{align}
    \mathbb{Z}^*_{p\cdot q}=\left\{[a]_{p\cdot q}|ggT(a,p\cdot q)=1\right\}
    [a]_{pq}^{|\mathbb{Z}^*_{p\cdot q}|+1}=[a]_{pq}
\end{align}
\subsection{Ordnung}\label{subsec:ordnung}
Die Ordnung $m$ eines Elementes $g\in M$ mit $e$ als neutralem Element ist:\\
\begin{align*}
    g^m=g\circ g \dots \circ g=e
\end{align*}
Beispielrechnung:\\
Gruppe $M=(\mathbb{Z}_{17}, \oplus)$, $g=4$ und dem neutralen Element $e=0$.
\begin{align*}
    [4]_{17}^m&=[0]_{17}\\
    4^m&\equiv_{17}0\\
    \\
    4^1&\equiv_{17}4\\
    4^2&\equiv_{17}8\\
    4^3&\equiv_{17}12\\
    4^4&\equiv_{17}16\\
    4^4&\equiv_{17}20\equiv_{17}3\\
    4^5&\equiv_{17}7\\
    4^6&\equiv_{17}11\\
    4^7&\equiv_{17}15\\
    4^8&\equiv_{17}19\equiv_{17}2\\
    4^9&\equiv_{17}6\\
    4^{10}&\equiv_{17}10\\
    4^{11}&\equiv_{17}14\\
    4^{12}&\equiv_{17}18\equiv_{17}1\\
    4^{13}&\equiv_{17}5\\
    4^{14}&\equiv_{17}9\\
    4^{15}&\equiv_{17}13\\
    4^{16}&\equiv_{17}17\equiv_{17}0\equiv_{17}4^m\\
    \\
    Ord(4)&=16\\
\end{align*}
Daher hat die 4 in der Gruppe $(\mathbb{Z}_{17}, \oplus)$ die Ordnung $16$.

    \section{Verschlüsselung}\label{sec:verschluesselung}
    \subsection{Ordnung von endlichen Gruppen}
Sei $(G, \circ)$ eine endliche Gruppe und $g\in G$.
\begin{enumerate}
    \item Dann gibt es ein $n\in \mathbb{N}$ mit $g^n=e$.
    Die kleinste solche Zahl $n$ wird als Ordnung von $g$ bezeichnet, Notation: $ord(g)\coloneqq n$.
    \item Es gilt $<g>=\left\{ g^k|k=0,\dots,ord(g)-1 \right\}$
\end{enumerate}
\begin{align}
    ord(g)=ord(<g>)=|<g>|
\end{align}
\subsection{Erzeugende Elemente}
Sei $m\in \mathbb{N}$ und $p\in \mathbb{P}$.
\begin{enumerate}
    \item Die Gruppe $(\mathbb{Z}_m, \oplus)$ ist zyklisch.
    Ein Element $[a]_m$ ist genau dann erzeugendes Element, wenn $ggT(a,m)=1$.
    \item Die Gruppe $(\mathbb{Z}_m^\times, \otimes)$ ist im allgemeinen nicht zyklisch.
    \begin{align}
        \mathbb{Z}_8^\times=\left\{ [1]_8, [3]_8, [5]_8, [7]_8\right\}
    \end{align}
    Es gilt $ord([1]_8)=1$ und $ord([a]_8)=2$ für $a=3,5,7$.
    \item Die Gruppe $(\mathbb{Z}_p^\times, \otimes)=(\mathbb{Z}_p \setminus \left\{ [0]_p\right\}, \otimes)$ ist zyklisch.
    Das bedeutet, \textbf{es gibt} ein multiplikativ erzeugendes Element von $(\mathbb{Z}_p\setminus \left\{ [0]_p \right\}, \otimes)$.
    Das Problem ist dann das finden eines solchen Elements.
\end{enumerate}

\subsection{Satz von Lagrange}
Sei $(G, \circ)$ eine endliche Gruppe und $H$ eine Untergruppe.
Dann gilt:
\begin{align*}
    ord(H)|ord(G)
\end{align*}
Da die Ordnung eines Gruppenelements $g$ gleich der Ordnung der von $g$ erzeugten Untergruppe $<g>$ ist, folgt:
\begin{align*}
    \forall g\in G: ord(g)|ord(G)
\end{align*}

\subsection{Kongruenz ist Äquivalenzrelation}
Sei $(G, \circ)$ eine Gruppe und $H$ eine Untergruppe.
\begin{enumerate}
    \item Dann ist $\equiv_H$ eine Äquivalenzrelation.
    \item Für die Äquivalenzklassen bezüglich dieser Äquivalenzrelation gilt:
    \begin{align*}
        \forall g\in G: [g]_{\equiv_{H}}=\left\{ g\circ h|h\in H\right\} \eqqcolon gH
    \end{align*}
\end{enumerate}

\subsection{Satz von Euler}
Sei $(G, \circ)$ eine endliche Gruppe und $g\in G$.
Dann gilt:
\begin{align*}
    g^{ord(G)}=e
\end{align*}
\\
Sei $m\in \mathbb{N}$.
Dann definieren wir
\begin{align*}
    \varphi(m)\coloneqq |\left\{ a\in \mathbb{N}|a\leq m \wedge ggT(a,m)=1 \right\}|
\end{align*}
als Anzahl der zu $m$ teilerfremden natürlichen Zahlen zwischen 1 und $m$.
$\varphi$ heißt die Eulersche $\varphi$-Funktion.
Es gilt:
\begin{align*}
    \varphi(m)=ord(\mathbb{Z}_m^{\times})
\end{align*}
\\
Seien $m, a\in\mathbb{N}$ mit $ggT(m,a)=1$.
Dann gilt:
\begin{align*}
    a^{\varphi(m)}\equiv_m 1
\end{align*}
\\
Fermat hat mit dem kleinen Satz von Fermat einen Spezialfall bewiesen:\\
Sei $p\in\mathbb{P}$ und $a\in\mathbb{N}$ mit $1\leq a<p$
Dann gilt:
\begin{align*}
    a^{p-1}\equiv_p 1
\end{align*}
Oder anders gesagt:
\begin{align*}
    [a]_p^{p-1}= [1]_p
\end{align*}
\begin{enumerate}
    \item Sei $p\in \mathbb{P}$.
    Dann gilt:
    \begin{align*}
        \varphi(p)=p-1
    \end{align*}
    \item Seien $p\in \mathbb{P}$ und $k \in \mathbb{N}$.
    Dann gilt:
    \begin{align*}
        \varphi(p^k)=p^k-p^{k-1}=p^k(1-\frac{1}{p})
    \end{align*}
    \item Seien $m,n\in \mathbb{N}$ und $ggT(m,n)=1$.
    Dann gilt:
    \begin{align*}
        \varphi(m\cdot n)=\varphi(m)\cdot \varphi(n)
    \end{align*}
    \item Seien $p,q\in \mathbb{P}$ mit $p\neq q$.
    Dann gilt:
    \begin{align*}
        \varphi(p\cdot q)=\varphi(p-1)\cdot \varphi(q-1)
    \end{align*}
    \item Sei $n=\prod_{j=1}^{r}p_j^{k_{j}}$ mit $r\in \mathbb{N}$, $p_j\in\mathbb{P}$ und $k_j\in \mathbb{N}_0$ für alle $j=1,\dots,r$.
    Dann gilt:
    \begin{align*}
        \varphi(n)=n\cdot \prod_{j=1}^{r}(1-\frac{1}{p_j})
    \end{align*}
\end{enumerate}


\end{document}
