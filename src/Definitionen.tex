%! Author = engel
%! Date = 04.06.2024

% Preamble
\documentclass[11pt]{article}

% Packages
\usepackage{amsmath}
\usepackage{amsfonts}
\usepackage{amssymb}

% Document
\begin{document}

    \section{Beschränkungen}\label{sec:beschrankungen}

    \subsection{Obere Schranke}\label{subsec:obere-schranke}
    \begin{align*}
        \exists S:\forall x\in M: x\leq S
    \end{align*}

    \subsection{Untere Schranke}\label{subsec:untere-schranke}
    \begin{align*}
        \exists S:\forall x\in M: x\geq S
    \end{align*}

    \subsection{Maximum}\label{subsec:maximum}
    Das Maximum $a$ ist genau dann ein Maximum der Menge $M$, wenn gilt:
    \begin{align*}
        a\in M: \forall x\in M: x\leq a
    \end{align*}
    Das heißt aber im Umkehrschluss, dass es nicht zwingend ein Maximum geben muss.
    So zum Beispiel $M=[-1,1)$, da die $1$ nicht in der Menge enthalten ist, und es keine Näherung zu $1$ gibt.

    \subsection{Minimum}\label{subsec:minimum}
    Das Minimum $a$ ist genau dann ein Minimum der Menge $M$, wenn gilt:
    \begin{align*}
        a\in M: \forall x\in M: x\geq a
    \end{align*}
    Das heißt aber im Umkehrschluss, dass es nicht zwingend ein Minimum geben muss.
    So zum Beispiel $M=(-1,1]$, da die $-1$ nicht in der Menge enthalten ist, und es keine Näherung zu $-1$ gibt.

    \subsection{Supremum}\label{subsec:supremum}
    Das Supremum ist die kleinste obere Grenze und damit der größte Punkt in der Menge $M$.\\
    Anders ausgedrückt, die kleinste obere Schranke.

    \subsection{Infimum}\label{subsec:infimum}
    Das Supremum ist die größte untere Grenze und damit der kleinste Punkt in der Menge $M$.\\
    Anders ausgedrückt, die größte untere Schranke.

    \subsection{Unterschied}\label{subsec:unterschied}
    Dabei gilt der Unterschied zwischen dem \textbf{Supremum} und dem \textbf{Maximum}, beziehungsweise dem \textbf{Infimum} und dem \textbf{Minimum}, dass
    das \textbf{Maxi-} und dem \textbf{Minimum} zwingen in der Menge $M$ enthalten sein müssen, während das \textbf{Supremum} und das \textbf{Infimum} gegebenenfalls auch außerhalb der Menge $M$ liegen können.

    \section{Gruppen}\label{sec:gruppen}
    Eine algebraische Struktur ($G$, $\circ$) heißt genau dann Gruppe, wenn folgende Axiome erfüllt sind:
\begin{enumerate}
    \item \textbf{Abgeschlossenheit:} Gruppe $G$ ist bezüglich $\circ$ abgeschlossen: $\forall a,b\in G : \exists c: G: a\circ b=c$
    \item \textbf{Assoziativgesetz:} Operation $\circ$ ist Assoziativ: $\forall a,b,c\in G: (a\circ b)\circ c=a\circ (b\circ c)$
    \item \textbf{Linksneutrales Element:} In $G$ gibt es ein linksneutrales Element: $\exists e\in G: \forall a\in G: a\circ e = a$
    \item \textbf{Inverses Element:} In $G$ gibt es zu jedem Element ein Inverses Element: $\forall a\in G:\exists a'\in G: a\circ a'=a'\circ a=e$
\end{enumerate}
Die Gruppe heißt zusätzlich Abel'sche Gruppe, wenn:
\begin{enumerate}
    \setcounter{enumi}{4}
    \item \textbf{Kommutativgesetz:} Operation $\circ$ ist Kommutativ: $\forall a,b\in M: a\circ b=b\circ a$
\end{enumerate}
\subsection{Elementare Gruppeneigenschaften}\label{subsec:elementare-gruppeneigenschaften}
\begin{enumerate}
    \item Sei $e\in G$ linksneutrales Element und $a,a'\in G$, sodass $a'\circ a=e$.
    Dann gilt auch $a\circ a'=e$.
    Man nennt $a'$ dann nur noch ein inverses zu $a$.
    \item Sei $e\in G$ linksneutrales Element.
    Dann gilt $a\circ e=a$ für alle $a\in G$ und man nennt $e$ dann ein neutrales Element.
    \item Es gibt genau ein neutrales Element $e\in G$.
    \item Zu jedem $a\in G$ gibt es genau ein inverses Element $a'\in G$ (oft mit $a^-1$ bezeichnet).
    \item $(a^{-1})^{-1}=a$.
    \item $(a\circ b)^{-1}=b^{-1}\circ a^{-1}$.
\end{enumerate}

\subsection{Gruppeneigenschaften nachweisen}\label{subsec:gruppeneigenschaften-nachweisen}
\begin{enumerate}
    \item Wählen wir $a''$ mit $a''\circ a'=e$.
    Dann gilt: $a\circ a'=e\circ a\circ a'=a''\circ a'\circ a\circ a'=a''\circ e\circ a'=a''\circ a'=e$
    \item $a\circ e=a\circ(a'\circ a)=(a\circ a')\circ a=e\circ a=a$
    \item Wir nehmen an, es gäbe zwei neutrale Elemente von $G$ ($e,e^*\in G$).
    Dann ist jedoch $e^*=e\circ e^*=e$.
    Die beiden neutralen Elemente sind also zwangsläufig gleich.
    \item Wir nehmen an, es gäbe ein zweites inverses Element $a^*$ neben $a'$ zu $a$ in $G$.
    Dann muss jedoch $a^*=a'$ sein: $a^*=a^*\circ e=a^*\circ (a\circ a')=(a^*\circ a)\circ a'=e\circ a'=a'$
    \item Es gilt $(a^{-1})^{-1}\circ a^{-1}=e$.
    Es gilt aber auch  $a\circ a^{-1}=e$.
    Zu $a^{-1}$ gibt es jedoch nur ein inverses Element (nach der vierten Eigenschaft der Gruppen).
    Also ist $(a^{-1})^{-1}=a$.
    \item Es gilt $(b^{-1}\circ a^{-1})\circ(a\circ b)=b^{-1}\circ(a^{-1}\circ a)\circ b=b^{-1}\circ e\circ b=b^{-1}\circ b=e$.
    Somit ist $(b^{-1}\circ a^{-1})$ ds inverse Element zu $a\circ b$.
    Also ist $(a\circ b)^{-1}=b^{-1}\circ a^{-1}$.
\end{enumerate}
\subsection{Ordnung}\label{subsec:ordnung}
Die Ordnung $m$ eines Elementes $g\in M$ mit $e$ als neutralem Element ist:\\
\begin{align*}
    g^m=g\circ g \dots \circ g=e
\end{align*}
Beispielrechnung:\\
Gruppe $M=(\mathbb{Z}_{17}, \oplus)$, $g=4$ und dem neutralen Element $e=0$.
\begin{align*}
    [4]_{17}^m&=[0]_{17}\\
    4^m&\equiv_{17}0\\
    \\
    4^1&\equiv_{17}4\\
    4^2&\equiv_{17}8\\
    4^3&\equiv_{17}12\\
    4^4&\equiv_{17}16\\
    4^4&\equiv_{17}20\equiv_{17}3\\
    4^5&\equiv_{17}7\\
    4^6&\equiv_{17}11\\
    4^7&\equiv_{17}15\\
    4^8&\equiv_{17}19\equiv_{17}2\\
    4^9&\equiv_{17}6\\
    4^{10}&\equiv_{17}10\\
    4^{11}&\equiv_{17}14\\
    4^{12}&\equiv_{17}18\equiv_{17}1\\
    4^{13}&\equiv_{17}5\\
    4^{14}&\equiv_{17}9\\
    4^{15}&\equiv_{17}13\\
    4^{16}&\equiv_{17}17\equiv_{17}0\equiv_{17}4^m\\
    \\
    Ord(4)&=16\\
\end{align*}
Daher hat die 4 in der Gruppe $(\mathbb{Z}_{17}, \oplus)$ die Ordnung $16$.

\end{document}
