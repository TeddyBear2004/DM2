%! Author = engel
%! Date = 02.06.2024

% Preamble
\documentclass[11pt]{article}

% Packages
\usepackage{amsmath}
\usepackage{amsfonts}

% Document
\begin{document}
    \section[5.3]{Aufgabe 5.3}\label{sec:aufgabe-5.3}
    Bestimmen Sie für $m=7$ und $m=10$ und die Zahlen $145, 200$ und $711$ jeweils die zugehörige Restklasse $[r]_{m}$ mit $0\leqr<m$\\
    \begin{align*}
    [145]
        _7&=[5]_7\\
        [145]_{10}&=[5]_{10}\\
        \\
        [200]_7&=[4]_{7}\\
        [200]_{10}&=[0]_{10}\\
        \\
        [711]_7&=[4]_{7}\\
        [711]_{10}&=[1]_{10}\\
    \end{align*}
    \section[5.3]{Aufgabe 5.3}\label{sec:aufgabe-5.32}
    Es seinen $m,n\in \mathbb{N}$.
    Zeigen Sie, dass die folgenden Aussagen äquivalent sind:\\
    \\
    $(1)$ $\forall a\in\mathbb{Z}:[a]_m\subseteq[a]_n$.\\
    $(2)$ $\exists a\in\mathbb{Z}:[a]_m\subseteq[a]_n$.\\
    $(3)$ $n|m$\\
    \\
    $(1)\Rightarrow(2)$: Es gelte $(1)$.
    Da $\mathbb{Z}\neq\emptyset$ ist, folgt dann auch $(2)$.\\
    \\
    $(2)\Rightarrow(3)$: Es gelte $(2)$.
    Wähle also $a\in\mathbb{Z}$ mit $[a]_m\subseteq[a]_n$.\\
    Daraus gilt: $[a]_m=a+m\mathbb{Z}$ und $[a]_n=a+n\mathbb{Z}$.\\
    Daher gilt auch: $a+m=a+m\cdot1\in[a]_m$ also auch $a+m=a+n\cdot k$.\\
    Es folgt $m=n\cdot k$, also $n|m$.
\end{document}