\subsection{Ordnung von endlichen Gruppen}
Sei $(G, \circ)$ eine endliche Gruppe und $g\in G$.
\begin{enumerate}
    \item Dann gibt es ein $n\in \mathbb{N}$ mit $g^n=e$.
    Die kleinste solche Zahl $n$ wird als Ordnung von $g$ bezeichnet, Notation: $ord(g)\coloneqq n$.
    \item Es gilt $<g>=\left\{ g^k|k=0,\dots,ord(g)-1 \right\}$
\end{enumerate}
\begin{align}
    ord(g)=ord(<g>)=|<g>|
\end{align}
\subsection{Erzeugende Elemente}
Sei $m\in \mathbb{N}$ und $p\in \mathbb{P}$.
\begin{enumerate}
    \item Die Gruppe $(\mathbb{Z}_m, \oplus)$ ist zyklisch.
    Ein Element $[a]_m$ ist genau dann erzeugendes Element, wenn $ggT(a,m)=1$.
    \item Die Gruppe $(\mathbb{Z}_m^\times, \otimes)$ ist im allgemeinen nicht zyklisch.
    \begin{align}
        \mathbb{Z}_8^\times=\left\{ [1]_8, [3]_8, [5]_8, [7]_8\right\}
    \end{align}
    Es gilt $ord([1]_8)=1$ und $ord([a]_8)=2$ für $a=3,5,7$.
    \item Die Gruppe $(\mathbb{Z}_p^\times, \otimes)=(\mathbb{Z}_p \setminus \left\{ [0]_p\right\}, \otimes)$ ist zyklisch.
    Das bedeutet, \textbf{es gibt} ein multiplikativ erzeugendes Element von $(\mathbb{Z}_p\setminus \left\{ [0]_p \right\}, \otimes)$.
    Das Problem ist dann das finden eines solchen Elements.
\end{enumerate}

\subsection{Satz von Lagrange}
Sei $(G, \circ)$ eine endliche Gruppe und $H$ eine Untergruppe.
Dann gilt:
\begin{align*}
    ord(H)|ord(G)
\end{align*}
Da die Ordnung eines Gruppenelements $g$ gleich der Ordnung der von $g$ erzeugten Untergruppe $<g>$ ist, folgt:
\begin{align*}
    \forall g\in G: ord(g)|ord(G)
\end{align*}

\subsection{Kongruenz ist Äquivalenzrelation}
Sei $(G, \circ)$ eine Gruppe und $H$ eine Untergruppe.
\begin{enumerate}
    \item Dann ist $\equiv_H$ eine Äquivalenzrelation.
    \item Für die Äquivalenzklassen bezüglich dieser Äquivalenzrelation gilt:
    \begin{align*}
        \forall g\in G: [g]_{\equiv_{H}}=\left\{ g\circ h|h\in H\right\} \eqqcolon gH
    \end{align*}
\end{enumerate}

\subsection{Satz von Euler}
Sei $(G, \circ)$ eine endliche Gruppe und $g\in G$.
Dann gilt:
\begin{align*}
    g^{ord(G)}=e
\end{align*}
\\
Sei $m\in \mathbb{N}$.
Dann definieren wir
\begin{align*}
    \varphi(m)\coloneqq |\left\{ a\in \mathbb{N}|a\leq m \wedge ggT(a,m)=1 \right\}|
\end{align*}
als Anzahl der zu $m$ teilerfremden natürlichen Zahlen zwischen 1 und $m$.
$\varphi$ heißt die Eulersche $\varphi$-Funktion.
Es gilt:
\begin{align*}
    \varphi(m)=ord(\mathbb{Z}_m^{\times})
\end{align*}
\\
Seien $m, a\in\mathbb{N}$ mit $ggT(m,a)=1$.
Dann gilt:
\begin{align*}
    a^{\varphi(m)}\equiv_m 1
\end{align*}
\\
Fermat hat mit dem kleinen Satz von Fermat einen Spezialfall bewiesen:\\
Sei $p\in\mathbb{P}$ und $a\in\mathbb{N}$ mit $1\leq a<p$
Dann gilt:
\begin{align*}
    a^{p-1}\equiv_p 1
\end{align*}
Oder anders gesagt:
\begin{align*}
    [a]_p^{p-1}= [1]_p
\end{align*}
\begin{enumerate}
    \item Sei $p\in \mathbb{P}$.
    Dann gilt:
    \begin{align*}
        \varphi(p)=p-1
    \end{align*}
    \item Seien $p\in \mathbb{P}$ und $k \in \mathbb{N}$.
    Dann gilt:
    \begin{align*}
        \varphi(p^k)=p^k-p^{k-1}=p^k(1-\frac{1}{p})
    \end{align*}
    \item Seien $m,n\in \mathbb{N}$ und $ggT(m,n)=1$.
    Dann gilt:
    \begin{align*}
        \varphi(m\cdot n)=\varphi(m)\cdot \varphi(n)
    \end{align*}
    \item Seien $p,q\in \mathbb{P}$ mit $p\neq q$.
    Dann gilt:
    \begin{align*}
        \varphi(p\cdot q)=\varphi(p-1)\cdot \varphi(q-1)
    \end{align*}
    \item Sei $n=\prod_{j=1}^{r}p_j^{k_{j}}$ mit $r\in \mathbb{N}$, $p_j\in\mathbb{P}$ und $k_j\in \mathbb{N}_0$ für alle $j=1,\dots,r$.
    Dann gilt:
    \begin{align*}
        \varphi(n)=n\cdot \prod_{j=1}^{r}(1-\frac{1}{p_j})
    \end{align*}
\end{enumerate}

\begin{align*}
    \varphi(p)&=p-1\\
    n&=pq\\
    \varphi(pq)&=pq-1-(q-1)-(p-1)\\
    &=(p-1)(q-1)\\
    \\
    n&=6=2\cdot3\\
    \varphi(6)&=1\cdot 2=2
\end{align*}
Satz von Lagrange:
\begin{align*}
    \forall a\in\mathbb{Z}_n^{\times}:a^{\varphi(n)}&\equiv_{n}1\\
    \forall a\in\mathbb{Z}_p^{\times}:a^{p-1}&\equiv_{p}a\\
    \forall a\in\mathbb{Z}_p:a^{(p-1)+1}&\equiv_{p}a\\
    n&=p\cdot q\\
    \varphi(n)&=(p-1)(q-1)\\
    \forall a\in\mathbb{Z}_{pq}^{\times}:a^{(p-1)(q-1)}&\equiv_{pq}1^\\
    \forall a\in\mathbb{Z}_{pq}:a^{(p-1)(q-1)+1}&\equiv_{pq}a\\
    \\
    (p-1)(q-1)+1&=e\cdot d\\
    ed&\equiv_{(p-1)(q-1)}1\\
    a^{e\cdot d}&\equiv_{pq}(a^e)^d\\
    &\equiv_{pq}(a^d)^e\\
\end{align*}
\begin{align*}
    p\coloneqq7\\
    q\coloneqq11\\
    e=13\\
    d=(p-1)(q-1)=6\cdot 10=60\\
    ggt(13,60)=1\\
    (n,e)=(77,13)\\
    [13]_{(p-1)(q-1)}\otimes[d]_{(p-1)(q-1)}=[1]_{(p-1)(q-1)}\\
    [13]_{60}\otimes[d]_{60}=[1]_{60}\\
    60&=4\cdot13+8=5\cdot(60-4\cdot13)-3\cdot13=5\cdot60-23\cdot13\\
    13&=1\cdot8+5\leftrightarrow2\cdot8-3)13-1\cdot8=5\cdot8-3\cdot13\\
    8&=1\cdot5+3\leftrightarrow2\cdot(7-1\cdot5)-1\cdot5=2\cdot8-3\cdot5\\
    5&=1\cdot3+2\leftrightarrow3-1\dots(5-13)=2\cdot3-1\cdot5\\
    3&=1\cdot1+1\leftrightarrow1=3-1\cdot2\\
    2&=2\cdot1+0\\
    \\
    d=37\text{ ist das mutliplikative Inverse zur 13, also:}\\
    [37]_{60}\otimes[13]_{60}=[1]_{60}
\end{align*}